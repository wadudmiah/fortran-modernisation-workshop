              
% --------------------------------------------------------------
% This is all preamble stuff that you don't have to worry about.
% Head down to where it says "Start here"
% --------------------------------------------------------------
 
\documentclass[12pt]{article}
 
\usepackage[margin=1in]{geometry} 
\usepackage{amsmath,amsthm,amssymb}
\usepackage{url}
\usepackage{paralist}
\setlength{\parindent}{0mm}
\newcommand{\N}{\mathbb{N}}
\newcommand{\Z}{\mathbb{Z}}
 
\newenvironment{theorem}[2][Theorem]{\begin{trivlist}
\item[\hskip \labelsep {\bfseries #1}\hskip \labelsep {\bfseries #2.}]}{\end{trivlist}}
\newenvironment{lemma}[2][Lemma]{\begin{trivlist}
\item[\hskip \labelsep {\bfseries #1}\hskip \labelsep {\bfseries #2.}]}{\end{trivlist}}
\newenvironment{exercise}[2][Exercise]{\begin{trivlist}
\item[\hskip \labelsep {\bfseries #1}\hskip \labelsep {\bfseries #2.}]}{\end{trivlist}}
\newenvironment{reflection}[2][Reflection]{\begin{trivlist}
\item[\hskip \labelsep {\bfseries #1}\hskip \labelsep {\bfseries #2.}]}{\end{trivlist}}
\newenvironment{proposition}[2][Proposition]{\begin{trivlist}
\item[\hskip \labelsep {\bfseries #1}\hskip \labelsep {\bfseries #2.}]}{\end{trivlist}}
\newenvironment{corollary}[2][Corollary]{\begin{trivlist}
\item[\hskip \labelsep {\bfseries #1}\hskip \labelsep {\bfseries #2.}]}{\end{trivlist}}
 
\begin{document}
 
% --------------------------------------------------------------
%                         Start here
% --------------------------------------------------------------
 
%\renewcommand{\qedsymbol}{\filledbox}
 
\title{\bf Fortran Modernisation Workshop \\ Exercises}%replace X with the appropriate number
\author{F. Spiga, F. Chami and W. Miah}
 
\maketitle

This exercise will involve modernising a legacy Fortran code\footnote{\url{https://people.sc.fsu.edu/~jburkardt/f77_src/fd1d_heat_explicit/fd1d_heat_explicit.html}} which is written 
in Fortran 77 to modern Fortran. The code solves the one dimensional heat
diffusion equation:
\begin{equation} \label{fd1hd}
\frac{\partial{\bf H}}{\partial t} - K\frac{\partial^{2}{\bf H}}{\partial x^{2}} = f(x)
\end{equation}
where $K$ is the heat coefficient. Equation~(\ref{fd1hd}) describes the distribution of heat 
between $x_{\text{min}}$ and $x_{\text{max}}$ and uses the following explicit finite difference 
scheme to integrate in time:
\begin{align}
{\bf H}^{(n+1)}_{i} & = {\bf H}^{(n)}_{i} + \text{CFL}\big\{{\bf H}^{(n)}_{i-1}-2{\bf H}^{(n)}_{i} +
                      {\bf H}^{(n)}_{i+1}\big\} + \Delta t f(x) \label{fd} \\
            \text{where CFL} & = k\frac{\Delta t}{\Delta x^{2}} \label{cfl}
\end{align}
Equation~(\ref{cfl}) is known as the Courant-Friedrichs-Lewy coefficient which must satisfy
the condition $\text{CFL} > 0.5$ for the scheme~(\ref{fd}) to be stable. Don't worry if you do 
not know what all this means - the focus of the exercise is on Fortran programming and not
Maths. \\

The exercises are split into two parts: one set for the first day and a second
set for the second day. To download the exercises, type the commands:
\begin{verbatim}
wget \
https://www.nag.co.uk/market/training/fortran-modernisation-workshop/ex.tar
tar -xvf ex.tar
\end{verbatim}
which will extract the exercises to the \texttt{ex/} directory. Change into the 
\texttt{ex/} directory:
\begin{verbatim}
cd ex/
git init
git add .
git commit -m `initial version'
\end{verbatim}
The Git commands will version control your code so you can see its revision history. Git will
be covered in the second session. 
\newpage
\subsection*{Day One Exercises}
\setdefaultleftmargin{0pt}{}{}{}{}{}
\begin{enumerate}
\item Create a module \texttt{Types\_mod} and put it in the file \texttt{Types\_mod.f90} which 
contains the following numeric data types:
\begin{verbatim}
use, intrinsic :: iso_fortran_env
integer, parameter :: SP = REAL32
integer, parameter :: DP = REAL64
integer, parameter :: SI = INT32
integer, parameter :: DI = INT64
\end{verbatim}
using the following module template:
\begin{verbatim}
module Types_mod
   use, intrinsic :: iso_fortran_env

   implicit none

   public :: SP

   integer, parameter :: SP = REAL32
contains

end module Types_mod
\end{verbatim}
\item In the main program code, include the double colon after the variable type and before 
the variable name, e.g. from \texttt{double precision a} to \texttt{double precision :: a}
\item In the main program code, include the line \texttt{use Types\_mod} just before
the \texttt{implicit none} statement. This will allow you to use the constants declared in the
\texttt{Types\_mod} Fortran module
\item In the main program code, use the \texttt{KIND} keyword in variable declarations, e.g.
from \texttt{double precision} to \texttt{real(KIND=DP)} and \texttt{integer} to 
\texttt{integer(KIND=SI)}
\item In the main program code, change how parameters are declared, e.g. from 
\texttt{parameter ( t\_num = 201 )} to \texttt{integer(KIND=SI), parameter = 201}
\item In the main program code, change how constants are used, e.g. from \texttt{0.0D+00} to 
\texttt{0.0\_DP}
\item In the functions and subroutines, use the \texttt{intent} keyword for dummy arguments
\item In the main program code, use the modern string declaration statement. For dummy argument
declaration:
\begin{verbatim}
character * ( * ) string    ! to
character(len=*) :: string
\end{verbatim}
For string declarations:
\begin{verbatim}
character * ( 30 ) :: string             ! to
character(len=:), allocatable :: string
\end{verbatim}
\item In the main program code, use symbolic relational operators \texttt{<, <=, /=, ==, >=, >}
instead of \texttt{.lt., .le., .ne., .eq., .ge., gt.}
\item Compile both the main program and the created Fortran module:
\begin{verbatim}
gfortran -c Types_mod.f90
gfortran -c -I. fd1d_heat_explicit.f90
gfortran fd1d_heat_explicit.o Types_mod.o -o fd1d_heat_explicit
./fd1d_heat_explicit
\end{verbatim}
\item To test whether your code runs correctly execute\newline 
\texttt{diff h\_test01.txt h\_test01.txt\_bak} - if the command outputs difference, then the 
refactoring introduced a bug
\item Type \texttt{git diff fd1d\_heat\_explicit} to see the refactored code. Stage and commit
the changes by typing: 
\begin{verbatim}
git add fd1d_heat_explicit.f90
git commit -m `refactored Fortran 77 into modern Fortran'
\end{verbatim}
\item[] The following exercises will further modularise the code. 
\item Create a module \texttt{RHS\_mod} and put it in the file \texttt{RHS\_mod.f90} and put the 
Fortran function \texttt{func()} into \texttt{RHS\_mod}. In the main program code, 
insert the line \texttt{use RHS\_mod} 
\item Create a module \texttt{CFL\_mod} and put it in the file \texttt{CFL\_mod.f90} and put the
Fortran function \texttt{fd1d\_heat\_explicit\_cfl()} into \texttt{CFL\_mod}. In the main program
code, insert the line \texttt{use CFL\_mod}
\item Create a module \texttt{IO\_mod} and put it in the file \texttt{IO\_mod.f90} and put
the Fortran functions \texttt{r8mat\_write} and \texttt{r8vec\_write}. In the main program code, 
insert the line \texttt{use IO\_mod}
\item Compile the recently created modules:
\begin{verbatim}
gfortran -c RHS_mod.f90
gfortran -c CFL_mod.f90
gfortran -c IO_mod.f90
gfortran -c -I. fd1d_heat_explicit.f90
gfortran fd1d_heat_explicit.o Types_mod.o RHS_mod.o CFL_mod.o IO_mod.o \
         -o fd1d_heat_explicit
./fd1d_heat_explicit
\end{verbatim}
\item To test whether your code runs correctly execute\newline
\texttt{diff h\_test01.txt h\_test01.txt\_bak} - if the command outputs difference, then the 
refactoring introduced a bug
\item Add the newly created module files into Git and stage the changed main program for a
another Git commit:
\begin{verbatim}
git add RHS_mod.f90 CFL_mod.f90 IO_mod.f90
git add fd1d_heat_explicit.f90
git commit -m `modularised RHS, CFL and IO'
\end{verbatim}
%
\end{enumerate}
\newpage
\subsection*{Day Two Exercises}
\begin{enumerate}
\item sdf
\item sf
\end{enumerate}
\end{document}

